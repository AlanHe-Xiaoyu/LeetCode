\documentclass{article}
\usepackage[left=1cm, right=1cm, top=1cm]{geometry}
\usepackage{amssymb}
\usepackage{amsmath}
\usepackage{color}
\usepackage{soul}
\usepackage{graphicx}
\usepackage{centernot}
\usepackage[dvipsnames]{xcolor}
\usepackage{mathrsfs}
\begin{document}

\newcommand\independent{\protect\mathpalette{\protect\independenT}{\perp}}
\def\independenT#1#2{\mathrel{\rlap{$#1#2$}\mkern2mu{#1#2}}}

% \noindent{\LARGE {\color{OrangeRed} Business and Financial Modeling Specialization }} \\


% \newpage
\noindent{\Large {\color{blue} FE RM }}
\begin{itemize}
\item Security pricing, Portfolio selection, Risk management

\item \textul{Defn} No-Arbitrage

For a contract : pay $p$ @ $t=0$, receive $c_k$ @ $t=k$, $k \in \mathbb{Z}^{+}$ ($c_k$ can be negative)

* Weak No-Arbitrage : $c_k \geq 0$ $\forall k$ $\implies$ $p \geq 0$

* Strong No-Arbitrage : $c_k \geq 0\ \forall k$ and $c_\ell > 0$ for some $\ell \implies p > 0$

\item Simple vs Compound interest

$A(1 + nr)$ vs $A(1+r)^{n}$

Continuous compounding : $\lim\limits_{k\rightarrow\infty} A(1 + \frac{r}{k})^{kn} = A \cdot e^{rn}$

Bond : for half-year bonds, yield to maturity $\lambda$ (= rate at which price $P$ = PV of coupon payments) follows
$$P = \frac{F}{(1 + \lambda/2)^{2T}} + \sum\limits_{k=1}^{2T} \frac{c}{(1 + \lambda/2)^k}$$

Lower quality $\rightarrow$ lower price $\rightarrow$ higher $\lambda$ (YTM) - crude measure however


\item \textul{Thm} {\color{red} Linear Pricing} : if price of CF $c_a$ is $p_a$ and price of CF $c_b$ is $p_b$, then price of CF that pays $c = c_a + c_b$ must be $p_a + p_b$


\item Floating interest rate : break down a floating $r$ bond into simpler CFs by constructing a deterministic CF

\begin{figure}[h!]
  \includegraphics[width=\linewidth]{/Users/ahecomputer/Desktop/All_Pics_LaTeX/coursera_floating_to_deterministic}
  % \caption{caption}
  % \label{label}
\end{figure}

Thus, the price of the portfolio above is $= p_k -\alpha + \alpha = \frac{F\cdot r_0}{(1 + r_0)^{k}}$

$$\implies P_f = \frac{F}{(1+r_0)^{n}} + \sum\limits_{k=1}^{n} p_k
 = \cdots = F
$$
Price of floating rate bond is always the face value!

\item Term structure of $r$, i.e. depends on duration

Spot rates $s_t =$ interest rate for a loan maturing in $t$ years

$A$ in year $t$ $\rightarrow \text{PV} = \frac{A}{(1+s_t)^{t}}$

Discount rate $d(0,t) = \frac{1}{(1+s_t)^{t}}$

Forward rate $f_{uv} =$ interest rate quoted today for lending from year $u$ to $v$ ($v \geq u$)

$$(1+s_v)^{v} = (1+s_u)^{u} \cdot (1 + f_{uv})^{v-u}
\implies
f_{uv} = \Big( \frac{(1+s_v)^{v}}{(1+s_u)^{u}} \Big)^{\frac{1}{v-u}} - 1
$$
Relation b/w spot and forward rates
$$(1+s_t)^{t} = \prod_{k=0}^{t-1} (1 + f_{k, k+1})
$$

\item \textul{Defn} Forward Contract : gives the buyer the right and obligation to purchase a specified amount of an asset, at a specified time $T$, at a specified price $F$ (called the forward price) set at time $t = 0$

$f_t$ = value/price at time $t$ of a long position in the forward contract

Value at time $T : f_T = S_T - F$ where $S_T$ is price of asset at $T$

Forward price $F$ is set s.t. at $t=0$ value/price $f_0 = 0$

\begin{figure}[h!]
  \includegraphics[width=\linewidth]{/Users/ahecomputer/Desktop/All_Pics_LaTeX/cousera_forward_price}
  % \caption{caption}
  % \label{label}
\end{figure}

$$0 = \Big(\frac{S_0}{d(0,T)} - F\Big) d(0,T) \implies F = \frac{S_0}{d(0,T)}$$

$F > S_0$ due to cost of carry

- $f_t = (F_t - F_0) \cdot d(t, T)$

\item \textul{Defn} Swap : contracts that transform one kind of cash flow into another

Leverage strength

$\rightarrow$ Pricing interest rate swaps (fixed rate vs floating - based on LIBOR)

CFs at $t = 1, ..., T$

A (long) receives $N r_{t-1}$, pays $NX$; B (short) receives $NX$, pays $N r_{t-1}$ where $X$ is the fixed rate.

Set value of swap $V_{A} = N(1 - d(0,T)) - NX \sum\limits_{t=1}^{T} d(0,t) = 0$ gives:

$$X = \frac{1 - d(0,T)}{\sum_{t=1}^{T} d(0,t)}$$


\item Futures

Need Martingale pricing formalism

Deterministic $r:$ forward price = futures price

At maturity, futures price $F_T$ = price of underlying $S_T$

\item \textul{Defn} European call/put option : gives the buyer the right but not the obligation to purchase/sell 1 unit of underlying at specified price $K$ (strike price) at a specified time $T$ (expiration).

\textul{Defn} American call/put option : gives ... right but not obligation to purchase/sell 1 unit ... at {\color{red} any time until} a specified $T$ (expiration).

- EU call option payoff = $\max\{S_T - K, 0\}$, i.e. nonlinear

Intrinsic value of a EU call option at time $t \leq T = \max\{S_t - K, 0\}$

$\rightarrow$ in the money if $S_t > K$; at the money if $S_t = K$; out of the money if $S_t < K$

- Notation

Price of EU call/put with strike $K$ and expiration $T$ : $c_E(t; K,T)$ and $p_E(t; K,T)$

Price of American call/put with strike $K$ and expiration $T$ : $c_A(t; K,T)$ and $p_A(t; K,T)$

{\color{red} European put-call parity} at time $t$ for non-dividend paying stock : $p_E(t;K,T) + S_t = c_E(t;K,T) + K \cdot d(t,T)$

With dividend (PV of all dividend until maturity = $D$), $p_E(t; K,T) + S_t - D = c_E(t; K,T) + K \cdot d(t,T)$

- Never optimal to exercise an American call on a non-dividend paying stock early!

Since $c_A(t;K,T) \geq c_E(t;K,T) \geq \max\{S_t - Kd(t,T), 0\} > \max\{S_t - K, 0\}$

Not always true for put options.

\item Options Pricing

Assume stock price dynamics in Binomial (each step Bernoulli($p$) goes up by $u$ portion or down by $d$ with $ud = 1$)

Assume a risk-free asset (or cash account) is available, i.e. \$1 becomes $(1+r)^{t}$ after $t$ periods

Utility function $u(\cdot)$ should be monotonic and concave, e.g. log

\end{itemize}














































\end{document}