\documentclass{article}
\usepackage[left=1cm, right=1cm, top=1cm]{geometry}
\usepackage{amssymb}
\usepackage{amsmath}
\usepackage{color}
\usepackage{soul}
\usepackage{graphicx}
\usepackage{centernot}
\usepackage[dvipsnames]{xcolor}
\usepackage{mathrsfs}
\begin{document}

\newcommand\independent{\protect\mathpalette{\protect\independenT}{\perp}}
\def\independenT#1#2{\mathrel{\rlap{$#1#2$}\mkern2mu{#1#2}}}


\noindent{\Large {\color{blue} Week 2 }}
\begin{itemize}
	\item Problems
	
	1. Inflation : consumer price index (CPI), production price idx (PPI), GDP deflator

	2. Unemployment : frictional (not much concern), cyclical (recession), structural (e.g. robots)

	3. Rate of Economic Growth : measured by (nominal vs real) GDP; flow of product (expenditures) and/or flow of cost (income)

	- GDP deflator = Nominal GDP / Real GDP

	\item Major macroeconomic policy tools

	Fiscal policy \& Monetary policy (often used together)

	Fiscal
	\begin{itemize}
		\item To stimulate the economy (fight recession) : increased government spending, tax cuts
		\item To contract (fight inflation) : decreased ..., tax hikes
	\end{itemize}

	Monetary
	\begin{itemize}
		\item To stimulate the economy (fight recession) : increase the money supply
		\item To contract (fight inflation) : decrease the money supply
	\end{itemize}

	Borrowers benefit from unexpected inflation.

	\item Aggregate Supply-Aggregate Demand (AS-AD) model


	\item Classical Economics Principles (rip in the 1929 Great Depression)

	Unemployment is a natural part of business cucle

	Economy is self-correcting

	No need for govt intervention like fiscal or monetary policy!

	\item Keynesian (came in at the GD, rip in 1970s stagflation)

	Keynesian spending cure - the New Deal (fiscal)

	Problem : lays the foundation of {\color{red} stagflation}, i.e. high inflation + high unemployment

	- Demand-Pull inflation : too much money chasing too few goods (Vietnam war and Johnson)

	- Cost-Push inflation : rapid increases in raw material prices or wage increases drive up production costs (can happen as a result of ``supply shock'')

	Example : The Kennedy Tax Cut of 1964 was a modern day version of Keynesian economics


	\item Monetarism

	Lowest sustainable unemployment rate (LSUR) or natural rate


	\item Supply Side Economics : lower taxes to lower budget deficit

	``New Classical'' $\rightarrow$ a complete rejection to Keynesians

	Theory of Rational Expectations : ppl will remember, so policies don't help (?)

	New century, globally start to face structural issues that are resistent to Keynesian soln's.

\end{itemize}




\noindent{\Large {\color{blue} Week 3 }}
\begin{itemize}
\item Two pillars of Classical Econ (crumbled under Keynesian)

Say's Law : supply creates its own demand

Quantity Theory of Money : PQ = MV (M = money supply; V = velocity of money, sim to interest rate; P = general price level as measured by an idx, e.g. CPI; Q = quantity of real output sold)

\item AS-AD : allows price adjustment; Keynesian : assume price fixed

Factors that shift AD (determinants of AD) : consumption, investment, gov't, net export

Factors that shift AS : input prices, technology and productivity, legal-institutional environment

\item Three Ranges of Economy (aggregate output vs price level) : Keynesian (horizontal), Intermediate, Classical (vertical)

\item Keynesian : price adjustment overwhelmed by ``income adjustment mechanism''
\end{itemize}





\noindent{\Large {\color{blue} Week 4 }}
\begin{itemize}
\item Keynesian model gave rise to Fiscal Policy (gov't expenditure + tax changes)

Keynesian Aggregate Production-Aggregate Expenditures (AP-AE) model


\item Leakages : consumer saving, business saving, taxes, imports

Injections : investment, gov't spending, exports

AE = $C + I + G + (X - M)$ where X = exports, M = imports $\rightarrow$ AE is the vertical sum of these four components


\item Autonomous consumption : consumption that occurs regardless of change in income (even if income = 0)

\item Marginal Propensity to Consume (MPC) : extra amount ppl consume when they receive an extra \$

Marginal Propensity to Save (MPS) = 1 - MPC

\item $C = C_0 + MPC \cdot Y_d$ where $Y_d$ is disposable income

AE not necessarily equal AP, so ``supply creates its own demand'' (Classical) is incorrect.

\item Investment Expenditures (15\% of total AE) : purchase of residential structures, investment in business plant and equipment (70\%), additions to company inventory

Assumption : independent with income (horizontal line)

Determinants of investment expenditures : interest rates (Keynes)

Gov't Expenditures (horizontal line)


\item \textul{Expansionary} fiscal policy : increase G and/or cut T to close a recessionary gap

\textul{Contractionary} fiscal policy : cut G and/or raise T to cool down overheated econ and curb inflation

\item Net Exports

Keynesian model (simplified) assumes a ``closed economy'' - no international trade


\item Keynesian (expenditure) Multiplier : a change in expenditure multiplied by this \# to determine the resulting change in total output

$>1$ since income is re-spent many times after the initial increase

Keynesian (expenditure) Multiplier = $\frac{1}{MPS} = \frac{1}{1 - MPC}$

Keynesian tax multiplier = MPC * Keynesian (expenditure) multiplier


\item Paradox of Thrift - consumer tries to save but actually saves less

Crowding out : a reduction in private sector investment that can be caused by increased gov't spending (The net effect of a fiscal policy stimulus may be less than intended).


\item Key weakness of Keynesian model : assumes price fixed + assumes away inflation; neglects the crucial influence of monetary factors on interest rates etc.


\begin{figure}[h!]
  \includegraphics[width=0.5\linewidth]{/Users/ahecomputer/Desktop/All_Pics_LaTeX/Keynesian_vs_Classical}
  % \caption{caption}
  % \label{label}
\end{figure}

\item Political mind plays a role in choosing to cut taxes or increase government expenditures to close a recessionary gap (conservative vs. liberal)
\end{itemize}








\newpage
\noindent{\Large {\color{blue} Week 5 }}
\begin{itemize}
\item Monetary policy

1969 to 1970s, \textul{stagflation} emerges, Monetarism challenges Keynesian

\item 3 kinds of money

Commodity money (gold nuggets, silver, grains, etc.)

Bank money (checkbooks and bank drafts)

Fiat or Paper money (currencies, e.g. \$, rmb)


\item Money : medium of exchange, standard of value, store of value

\item M1 (transactions money) = cash + checking accounts + traveler's checks

M2 (broad money) = M1 + saving accounts + time deposits + money market mutual funds

M3 = M1 + M2 + short term fiscal assets

\item Nominal vs Real interest rates


\item 2 major determinants of money demand : transactions demand \& asset demand

The basic determinant of transactions demand is Nominal GDP

\item ``Fractional'' reserve $\rightarrow$ money ``supply'' multiplier = 1 / RR, i.e. required reserve ratio

Bank Run : everyone comes in at once to demand money (fear becomes a self-fulfilling prophecy) $\rightarrow$ Fed Reserve is ``lender of last resort'', i.e. bankers' bank

Fed's most important tool : Open Market operations, e.g. buying (or selling) gov't securities to expand (or reduce) money supply

5-step monetary policy sequence called \textul{monetary transmission mechanism}

\item Keynesians believe that monetary policy is effective only for fine-tuning in mild recession/inflation.

Monetary believes that inflation happens when gov't prints too much money.

\item Stagflation = recession + inflation (increase in price)
\end{itemize}





\newpage
\noindent{\Large {\color{blue} Week 6 }}
\begin{itemize}
\item Okun's Law : GDP falls by every 2\%, then unemployment rises by 1\%

$\rightarrow$ impliest that \textul{actual} GDP must grow as rapidly as \textul{potential} GDP to keep the same unemployment rate.

\item Unanticipated inflation benefit borrowers at the expense of lenders.

Demand-Pull inflation vs Cost-Push inflation

\item Core or Inertial rate of inflation : ``inflationary expectations'' and a behavioral model (adaptive expectations),

Adaptive expectations assumes that people believe a constant (same) inflation rate next year $\rightarrow$ self-fulfilling prophecy


\item Phillips Curve (PC) : wages rose when unemployment was low (wages fell when unemployment was high)

Keynesians believe; Monetarists don't $\rightarrow$ they think $\exists$ natural rate of unemployment

Short run (PC curve) vs Long run (PC vertical line) $\rightarrow$ inflationary spiral

\item Supply Side economics : to cut taxes while increase tax revenues + accelerate growth w/o inducing inflation

Laffer Curve : tax revenues (x) vs marginal tax rate (y), a backward bending curve $\rightarrow$ at a certain pt, ppl will work less the more they're taxed.

According to SS economics, deregulation will shift out AS curve, which will then decrease inflation + increase GDP growth

\item ``New Classical'' economists replaced ``adaptive expectations'' with a new thoery of ``rational expectations'', rejecting short-run Keynesian solutions.
\end{itemize}







\newpage
\noindent{\Large {\color{blue} Week 7 }}
\begin{itemize}
\item Rational Expectations : assumes that ppl take into account {\color{red} all} available information

$\rightarrow$ implies that Keynesian policies are obsolete and useless

\item Rule of thumb : about 2 MIL jobs lost for every 1\% of GDP growth lost

\item Great Recession of 2007

Largest fiscal stimulus : Quantitative Easing (QE), i.e. Fed buys long-term bonds to drive up bond prices and thereby drive down yields and long-term interest rates.

\item 2000s Structural shift

Leading indicators : consumer confidences, ISM manufacturing idx, CPI, etc.

Trade deficits make GDP lower; offshoring decreases I within GDP eq.

$\rightarrow$ global butterfly effect

Keynesian can't deal with long term structural issues

\item What caused instability?

Keynesians believe instability arises from (1) changes in investment and consumption shift AD curve in/out, causing recession/inflation; and (2) adverse supply side shocks shift AS curve in, causing stagflation.

Monetarist believe that bad gov't policies are the major cause. V (velocity of money) is stable, while Keynesians believe it's unstable.

\item Self-Correcting?

Monetarist \& New Classical believe that internal mechanisms automatically move unemployment back; associated with adaptive and rational expectations.

Speed of adjustment?

Monetarist (adaptive expectations) : gradually, 2-3 years

New Classical (rational expectations) : fast, since future is fully anticipated

\item Balanced budget rule \& Crowding out (activist expansionary fiscal policy)

Treasury must raise interest rate to attract funds when selling bonds, which - due to ``zero sum game'' - would decrease spending on private investment

Monetarists believe it's a problem; Keynesians don't think so since business borrowing is usually depressed in a recession.

Keynesians oppose balanced budget rule since it would require contractionary fiscal policy during a recession and deepen it.

Supply Siders believe taxes and regulations must be reduced to get more output w/o inflation, so they favor discretionary policy (like Keynesians).


\item Monetarists believe that macroeconomic instability arises from \textul{bad gov't policies}

\end{itemize}










\newpage
\noindent{\Large {\color{blue} Week 8 }}
\begin{itemize}
\item Economic growth = expansion of otential GDP or national output

\item 4 supply factors of growth

(1) Human resources (quantity + quality) : labor, education, discipline, motivation

(2) Natural resoruces : land, minerals, fuels, environment

(3) Capital formation : machines, factories, roads; gov't projects are also key

(4) Technology : science, engineering, management, entrepreneurship

\item Demand factor : AD must grow

Efficiency factor : productive efficiency (least costly way of using resources) + allocative efficiency (max social well-being)

\item Growth Model

(1) Early growth theory (Adam Smith \& Thomas Malthus) : critical role of land $\rightarrow$ has to end/slow down due to Law of Diminishing Returns (LoDR)

Technological innovations and capital investment can overcome the LoDR $\rightarrow$ land is not the limiting factor

(2) Neoclassical Growth Model (Robert Solow) : capital \& technological change; primary tool : Aggregate Production Function (APF)

Capital deepening : increasing amount of capital per worker

LoDR applies here (not necessarily to land) in absence of tech change

$\rightarrow$ only thru tech change, we can avoid economic stagnation

\item Growth permits ``making a better living'', but doesn't guarantee ``good life''.

GDP per capita : essential measure of growth

\end{itemize}










\newpage
\noindent{\Large {\color{blue} Week 9 }}
\begin{itemize}
\item Budget Deficits - both agree chronic is bad

Keynesians believe they're a necessary byproduct of an expansionary fiscal policy during recessions.

Classicals believe they should be avoided except in wartime

\item Dark side of discretionary fiscal policy : budget deficits and public debt can occur when gov't use Keynesian fiscal policy to boost AD

Keynesian economics : fiscal stimulus and fiscal constraint

\item Budget Surplus : taxes and other revenues $>$ gov't expenditures this year

Balanced Budget

\item Issue {\color{red} bonds} to finance a budget deficit

National debt : total value of bonds owned by public

Debt calculated : the accumulated budget deficit $-$ accumulated surpluses

\item Debt-to-GDP ratio : very good yardstick of comparison

Real Versus Nominal Deficit : measures effective burden; real deficit = nominal deficit $-$ inflation rate $\cdot$ total debt

$\rightarrow$ gov't can lower debt by increasing inflation!

Structural vs Cyclical deficits : structural is about tax structures etc, cyclical is attributable to recession (partly due to auto stabilizers)

\item Okun's Law (again) : 1\% fall in unemployment $\iff$ 2\% rise in GDP
\item Calculation of structural deficit is sensitive to natural rate of unemployment (assumed).

Keynesian stimulus package would fail and create a larger deficit if the natural rate of unemployment is higher than assumed.

\item Finance a budget deficit : (1) raise taxes, (2) borrow money, (3) print money

(1) Raise taxes : good, but rarely used (politically unpopular)

Balanced budget multiplier (always 1) : when you simultaneously increase gov't expenditures and increases taxes by the same amount, you get an economic expansion exactly equal to the increase in gov't expenditures.

(2) Borrow money : US Treasury (Fed not involved) issue bonds or treasury bills; ``crowding out'' due to \textul{zero sum game}

Crowding out only applies to structural deficits $\rightarrow$ there's no automatic link b/w deficits and investment.

(3) Print money : Fed ``accommodates'' Treasury's expansionary fiscal policy, i.e. Fed buys the securities by printing \$

$\rightarrow$ increase in M can cause inflation

\item Deficit hawks vs. deficit doves (relatively harmless)

Hawks : chronic budget deficits crowd out private investment, increase trade deficits, and reduce economic growth + burden on future generations + redistributes income from poor to rich

Doves : budget deficits are stimulus to economic growth;
most debt is internal (owed to citizens); external \% has quadruples

$\rightarrow$ paying interest on external debt $\sim$ tax on US citizens by foreigners.

Hawks argue that productivity falls since gov't expenditures are on ``wasting assets'' (tanks, fighter jets, welfare program, etc.) rather than productive capital

\item A (potential) balanced budget amendment would compel Congress to annually balance its budget $\rightarrow$ impossible to use discretionary fiscal policy
\end{itemize}











\newpage
\noindent{\Large {\color{blue} Week 10 }}
\begin{itemize}
\item Absolute advantage : can produce a good at an absolute lower cost

{\color{red} Comparative advantage} : each country will benefit if it specializes in the export of goods that it can produce at \textul{relatively low} cost

$\rightarrow$ David Ricardo uses this to create the Free Trade model

\item Trade Barriers \& Protectionism (quota usually preferred politically over tariff)

Deadweight loss = loss in consumer surplus + loss in producer surplus

\item Dumping : occurs when foreign producers sell exports below cost of production

Retaliatory measures can cause trade wars, e.g. Smoot-Hawley Tariff Act of 1930

\item Non-tariff barriers (NTBs) : e.g. restrict the import of vegetables grown where certain pesticide is used (while knowing that other countries use this pesticide)

\item Problem of Ricandian Free Trade

A cheating country uses unfair trade practices to illegally subsidize exports $\rightarrow$ becomes a zero sum game
\end{itemize}










\newpage
\noindent{\Large {\color{blue} Week 11 }}
\begin{itemize}
\item Trade Identity Equation : if a country runs a trade deficit in its \textul{current account}, it must balance that with inflows into its \textul{capital account}.

% Current account

% Capital account

\item Exchange rate : currency trade rate

Appreciate : a country's currency that gains in value relative to another

Depreciate ...

5 reasons exchange rates move :

(1) diff rates of GDP growths - faster growth = depreciate

(2) diff rates of inflation - higher inflation rate = appreciate

(3) a change in real relative interest rates - higher $r$ = appreciate

(4) a change in tastes

(5) currency speculation.


\item Floating exchange rate system vs fixed exchange rate sysmte (``pegs'' its currency to the value of another or more)

\textul{Gold standard} : the currency issued had to either be gold or be redeemable in gold; if trade deficit, then had to use gold reserves to buy value of currency from falling

Gold specie flow mechanism (reason why gold standard popular) by Hume

After WW 1, many nations temporarily abandoned it to finance $\rightarrow$ inflation \& differing rates of inflation

$\rightarrow$ gold standard's collpase $\rightarrow$ desperate countries engage in ``competitive devaluations'', which eventually led to WW 2

$\rightarrow$ Bretton Woods agreement (\textul{partially fixed rate system} that could be periodically adjusted to reflect changes in currency values)

Dollar standard : \$ became world's key currency

1971, Nixon abandoned the dollar standard and Bretton Woods (dollars no longer redeemable for gold)

\item Hybrid system (today)

Moves to a \textul{managed float}, i.e. country buys or sells its currency to reduce day-to-day volatility of currency fluctuations

Also have the pegged \& currency blocs (e.g. Euro)

Allows monetary intervention

\item Multiplier Link : a contractionary policy in US might lead to contraction in EU as well.

Monetary Link
\item Germany (after WW 1) leads to EU-wide recession

Lesson : a country can't simultaneously have fixed exchange rates, open capital markets, and an independent monetary policy.

$\rightarrow$ EU creates common currency, Euro.
\end{itemize}













\newpage
\noindent{\Large {\color{blue} Week 12 }}
\begin{itemize}
\item Four elements in development : natural resources, human resources, capital formation, technology

\item Cycle of Poverty

Rapid population growth $\rightarrow$ low per capita income $\rightarrow$ low level of saving + low level of demand $\rightarrow$ low level of investment in physical and human capital $\rightarrow$ low productivity $\rightarrow$ (again) low per capita income

\item Policies for promoting growth

1. Establishing the rule of law

2. Opening economies to international trade

3. Controlling population growth

4. Encouraging foreign direct investment

5. Building human capital

6. Making peace with neighbors

7. Establishing independent central banks (to prevent hyper-inflation)

8. Establishing realistic exchange-rate policies

9. Privatizing state industries (to improve efficiency and entrepreneurship)
\end{itemize}














































\end{document}